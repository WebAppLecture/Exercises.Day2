\RequiredExercise{Taschenrechner basteln}
%
\par Sie bauen nun einen Taschenrechner mit HTML und JavaScript. Denkbar wäre die Verwendung eines Formulars für die Eingabedaten. Es soll zwei Textfelder geben, die durch eine \htag{select}-Box getrennt sind. Durch einen Button wird die Rechnung gestartet. Ein mögliches Design ist in Abb. \ref{fig:calculator} dargestellt.
%
\begin{figure}[!h]
\centering
\includegraphics{Figures/calculator.png}
\caption{Mögliches Design des Taschenrechners}
\label{fig:calculator}
\end{figure}
%
\par Die Operation soll durch eine \jkey{switch}-\jkey{case}-Anweisung entschieden werden. Nach der Berechnung soll das Ergebnis entweder in einer dritten (\jkey{readonly} gesetzten) Textbox oder in einem \htag{output}-Tag angezeigt werden.