\RequiredExercise{Ein Formular erschaffen}
%
\par Das Formular verfügt über 8 Felder, von denen Sie so oft wie möglich ein HTML5 Feld verwenden sollen:
%
\begin{itemize}
\item Name*, hier soll nach dem Muster Name, Vorname eingegeben werden, z.B.:
\begin{lstlisting}
[a-zA-Z\-]+,\s[a-zA-Z]+
\end{lstlisting}
\item E-Mail*
\item Telefonnummer, mit dem Format (für z.B. +49 (941) 1234):
\begin{lstlisting}
\+[0-9]{2}\s?\([0-9]{3}\)\s?[0-9]{4,}
\end{lstlisting}
\item Geburtstag* (Datum\footnote{Geht nicht bei Firefox (Version 7) und IE (Version 9)})
\item Anzahl der bisherigen Leistungspunkte (dies soll als Slider eingebaut werden von $0$ bis $180$)
\item Url der Homepage
\item Passwort*
\item Passwortbestätigung*
\end{itemize}
%
\par Mit * versehene Punkte sollen Pflichtfelder darstellen. Beim Betreten der Seite muss das erste Feld (Name) automatisch fokussiert werden. Einen Button zum Versenden des Formulars können Sie z.B. über \htag{input type=submit value='Ich bin der Button!'} erstellen.