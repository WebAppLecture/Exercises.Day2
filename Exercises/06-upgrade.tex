\RequiredExercise{Eine Webseite aktualisieren}
%
\par Laden Sie die Webseite auf
\url{https://github.com/WebAppLecture/Exercises.Day2/blob/master/Templates/Aufgabe6.html} herunter. Modifizieren Sie
diese so, dass die neue Seite den HTML5 Standards gerecht wird. Gehen Sie dazu
folgendermaßen vor:
%
\begin{itemize}
\item
Ersetzen Sie den Document Type durch \htag{!DOCTYPE html}
\item
Ersetzen Sie die \htag{div}s durch Ihre entsprechenden HTML5 Pendants – z.B.
\htag{div id=header} durch \htag{header} (Navigation sollte ebenfalls in den
entsprechenden Tag)
\item
Kürzen Sie unnötigen Code heraus – so wird aus dem \htag{meta}-Tag \htag{meta
\allowbreak charset=utf-8}
\item
Ändern Sie die \htag{div}s mit den post Klassen in entsprechende \htag{article}
Tags um
\item
Den Inhalt des Style Tags (über CSS werden Sie später mehr lernen) können Sie
durch die folgenden Modifikationen auf den neuesten Stand bringen:
%
\begin{lstlisting}
nav { float: left; width: 20%; }
article { float: right; width: 79%; }
footer { clear: both; }
\end{lstlisting}
%
\end{itemize}