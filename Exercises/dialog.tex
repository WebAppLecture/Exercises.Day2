\Exercise{Eigener Dialog mit Rückgabewert}
%
\par Beim Klicken eines Buttons soll ein Dialog (d.h. ein neues Browser-Fenster) über \jfunc{window.open} geöffnet werden, der eine TextBox mit der Aufforderung Text einzugeben beinhalten soll. In diesem Dialog soll bei Änderung des Eingabetextes immer, d.h. bei Eingabe, eine Auswertung dieses Textes erfolgen.
%
\par Am Besten implementieren Sie die für die Auswertung notwendigen Funktionen bereits am Anfang so, dass jede Funktion eine Variable (einen String) als Übergabewert erhält und einen Wert zurückgibt.
%
\par Die Auswertung soll direkt unter der TextBox angezeigt werden und folgende Daten umfassen:
%
\begin{itemize}
\item Länge des eingegebenen Textes
\item Anzahl der unterschiedlichen Zeichen (ohne Leerzeichen)
\item Anzahl der Wörter
\item Anzahl der Sonderzeichen und Zahlen (bzw. Nichtbuchstaben (\jvar{a-z}, \jvar{A-Z}))
\end{itemize}
%
\par Im Hauptdialog soll ebenfalls eine kleine Zusammenfassung angezeigt werden:
%
\begin{itemize}
\item Alle gefundenen Wörter (jedes in eine neue Zeile)
\item Summe aller gefundenen Zahlen (z.B. ist dies bei ``ha22llo12'' als Ergebnis 7)
\end{itemize}
%
\par Solang der Dialog geöffnet ist soll kein weiterer Dialog aufgemacht werden können (Tipp: Hierfür die Eigenschaft \jvar{closed} des \jvar{window} Objektes verwenden!).
%
\Remark{Weiterhin könnten Ihnen folgende Tipps sehr hilfreich sein:}
%
\begin{itemize}
\item Das Ereignis \jvar{onkeyup} eignet sich besser zum Überprüfen auf Veränderung als onchange, da \jvar{onchange} erst beim Verlassen (\jvar{onblur} Ereignis) des Eingabefeldes ausgelöst wird.
\item Um das neue Fenster als kleinen Dialog zu öffnen sollte folgender Code verwendet werden:
%
\begin{lstlisting}
window.open('','','width=300,height=500');
\end{lstlisting}
%
\item Über \jfunc{createElement} kann man das Element direkt als Variable erhalten
\item Die Ausgabe kann man über die innerHTML Eigenschaft von Elementen setzen – damit man nicht lauter Strings verketten muss kann man ein Array von Strings namens ausgabe anlegen, welches man anschließend über \jvar{ausgabe.join('<br/>')} so verbindet, dass man lauter einzelne Zeilen erhält.
\end{itemize}
%